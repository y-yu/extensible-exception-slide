\definecolor{links}{HTML}{2A1B81}
\hypersetup{colorlinks,linkcolor=,urlcolor=links}

\usetheme{Boadilla}
\usecolortheme{seahorse}
\usefonttheme{serif}
\beamertemplatenavigationsymbolsempty

\usepackage{luacode}
\usepackage{luatexja}
\usepackage{pgfpages}
\usepackage[osf]{mathpazo}

\begin{luacode*}
  USE_IPAFONT = os.getenv"USE_IPAFONT"
  USE_YUFONT = os.getenv"USE_YUFONT"
  
  if USE_YUFONT == "true" then
    tex.sprint("\\AtBeginDocument{\\usepackage[yu-osx, deluxe, expert]{luatexja-preset}}")
    tex.sprint("\\AtBeginDocument{\\setmainjfont[BoldFont=YuGothic Bold]{YuGothic Medium}}")
  elseif USE_IPAFONT == "true" then
    tex.sprint("\\AtBeginDocument{\\usepackage[ipaex, deluxe, expert]{luatexja-preset}}")
  else
    tex.sprint("\\AtBeginDocument{\\usepackage[hiragino-pro, deluxe, expert]{luatexja-preset}}")
    tex.sprint("\\AtBeginDocument{\\setmainjfont[BoldFont=Hiragino Kaku Gothic Pro W6]{Hiragino Kaku Gothic Pro W3}}")
  end
\end{luacode*}

\usepackage{epigraph}
\usepackage{etoolbox}
\usepackage{tikz}
\usepackage{framed}
\usepackage{libertine}
\usepackage[final]{listings}
\usepackage{amsmath}
\usepackage{mathtools}

%\setbeameroption{show notes on second screen=right}

%\usetikzlibrary{arrows,automata,shapes,backgrounds}

\setmainfont[Numbers=OldStyle, BoldFont=Palatino Bold]{Palatino}
\setsansfont{CMU Sans Serif}
\setmonofont{CMU Typewriter Text}

\title[Extensible Exception]{%
  \href{https://github.com/y-yu/extensible-exception-slide}{Extensible Exception} \\
  \href{http://kbkz.connpass.com/event/32420/}{\normalsize kbkz.tech \#10}
}
\author{吉村 優}
\date{July 16, 2016}
\institute[\url{https://twitter.com/\_yyu\_}]{%
  \url{https://twitter.com/\_yyu\_}\\
  \url{http://qiita.com/yyu}\\
  \url{https://github.com/y-yu}\\
}

\input{./lib/quotebox.tex}
\input{./lib/listings_setting.tex}
\input{./lib/footnotemark.tex}
\input{./lib/mydescription.tex}
\input{./lib/ballon.tex}

\newcommand\ballref[1]{%
\tikz \node[circle, shade,ball color=structure.fg,inner sep=0pt,%
  text width=8pt,font=\tiny,align=center] {\color{white}\ref{#1}};
}

%\everymath{\displaystyle}

\begin{document}

\frame{\maketitle}

\section{自己紹介}
\begin{frame}[fragile]
  \frametitle{自己紹介}
  
  \begin{columns}
    \begin{column}{0.4\textwidth}
      \centering
      \begin{figure}
        \includegraphics[width=0.9\textwidth]{img/bird2x.png}
      \end{figure}
    \end{column}
    \begin{column}{0.6\textwidth}
      \begin{itemize}
        \item<2-> \href{https://www.coins.tsukuba.ac.jp/}{筑波大学 情報科学類} 学士(COINS11)
        \item<3-> 現在はScalaを書く仕事に従事
      \end{itemize}
    \end{column}
  \end{columns}
\end{frame}

\section{エラー値とは?}
\begin{frame}[fragile]
  \frametitle{エラー値とは?}

  \begin{itemize}
    \item<2-> エラーであることを表す値
    \item<3-> 継承を使って次のように作るのが一般的
  \end{itemize}

  \uncover<4-> {
    \begin{tikzpicture}[sibling distance=10em,
      every node/.style = {shape=rectangle, rounded corners,
        draw, align=center,
        top color=white, bottom color=blue!20}]]
      \node {RootException}
      child { node {DBException} }
      child { node {FileException}
        child { node {ReadException} }
        child { node {WriteException} } }
      child { node {HttpException} };
    \end{tikzpicture}
  }

  \begin{center}
    \uncover<5->{
      \begin{tikzpicture}
        \calloutquote[author={},width=7cm,position={(-1.2,-0.5)},fill=green!30,rounded corners]{\textbf{後から}この階層構造を変更できる?}
      \end{tikzpicture}
    }
  \end{center}
\end{frame}

\begin{frame}
  \centering
  {\Huge Thank you for listening!}

  \quad

  {\Huge Any question?}
\end{frame}

\end{document}
