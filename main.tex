\definecolor{links}{HTML}{2A1B81}
\hypersetup{colorlinks,linkcolor=,urlcolor=links}

\usetheme{Boadilla}
\usecolortheme{seahorse}
\usefonttheme{serif}
\beamertemplatenavigationsymbolsempty

\usepackage{luacode}
\usepackage{luatexja}
\usepackage{pgfpages}
\usepackage[osf]{mathpazo}

\begin{luacode*}
  USE_IPAFONT = os.getenv"USE_IPAFONT"
  USE_YUFONT = os.getenv"USE_YUFONT"
  
  if USE_YUFONT == "true" then
    tex.sprint("\\AtBeginDocument{\\usepackage[yu-osx, deluxe, expert]{luatexja-preset}}")
    tex.sprint("\\AtBeginDocument{\\setmainjfont[BoldFont=YuGothic Bold]{YuGothic Medium}}")
  elseif USE_IPAFONT == "true" then
    tex.sprint("\\AtBeginDocument{\\usepackage[ipaex, deluxe, expert]{luatexja-preset}}")
  else
    tex.sprint("\\AtBeginDocument{\\usepackage[hiragino-pro, deluxe, expert]{luatexja-preset}}")
    tex.sprint("\\AtBeginDocument{\\setmainjfont[BoldFont=HiraKakuPro-W6]{HiraKakuPro-W3}}")
  end
\end{luacode*}

\usepackage{epigraph}
\usepackage{etoolbox}
\usepackage{tikz}
\usepackage{framed}
\usepackage{libertine}
\usepackage[final]{listings}
\usepackage{amsmath}
\usepackage{mathtools}
\usepackage{tikz-qtree}

%\setbeameroption{show notes on second screen=right}

%\usetikzlibrary{arrows,automata,shapes,backgrounds}

\setmainfont[Numbers=OldStyle, BoldFont=Palatino Bold]{Palatino}
\setsansfont{CMU Sans Serif}
\setmonofont{CMU Typewriter Text}

\title[Extensible Exception]{%
  \href{https://github.com/y-yu/extensible-exception-slide}{Extensible Exception} \\
  \href{http://kbkz.connpass.com/event/32420/}{\normalsize kbkz.tech \#10}
}
\author{吉村 優}
\date{July 16, 2016}
\institute[\url{https://twitter.com/\_yyu\_}]{%
  \url{https://twitter.com/\_yyu\_}\\
  \url{http://qiita.com/yyu}\\
  \url{https://github.com/y-yu}\\
}

\input{./lib/quotebox.tex}
\input{./lib/listings_setting.tex}
\input{./lib/footnotemark.tex}
\input{./lib/mydescription.tex}
\input{./lib/ballon.tex}

\newcommand\ballref[1]{%
\tikz \node[circle, shade,ball color=structure.fg,inner sep=0pt,%
  text width=8pt,font=\tiny,align=center] {\color{white}\ref{#1}};
}

%\everymath{\displaystyle}

\begin{document}

\frame{\maketitle}

\section{自己紹介}
\begin{frame}[fragile]
  \frametitle{自己紹介}
  
  \begin{columns}
    \begin{column}{0.4\textwidth}
      \centering
      \begin{figure}
        \includegraphics[width=0.9\textwidth]{img/bird2x.png}
      \end{figure}
    \end{column}
    \begin{column}{0.6\textwidth}
      \begin{itemize}
        \item<2-> \href{https://www.coins.tsukuba.ac.jp/}{筑波大学 情報科学類} 学士(COINS11)
        \item<3-> 現在はScalaを書く仕事に従事
        \item<4-> エラー処理に関する話をします
      \end{itemize}
    \end{column}
  \end{columns}
\end{frame}

\section{エラー値とは?}

\subsection{エラー値の階層構造}
\begin{frame}[fragile]
  \frametitle{エラー値とは?}

  \uncover<2-> {
    \begin{block}{エラー値}
      \begin{itemize}
        \item<3-> エラーであることを表す値
        \item<4-> 階層構造(木構造)になるのが一般的
      \end{itemize}
    \end{block}
  }

  \uncover<5-> {
    \begin{exampleblock}{エラー値の階層構造の例}
      \begin{center}
        \begin{tikzpicture}[scale=.8]
          \Tree[.RootException
                    HttpException 
                    DbException 
                    [.FileException 
                        ReadException
                        WriteException ] ]
        \end{tikzpicture}
      \end{center}
    \end{exampleblock}
  }

  \begin{center}
    \uncover<6->{
      \begin{tikzpicture}
        \calloutquote[author={},width=8cm,position={(-1.2,-0.2)},fill=green!30,rounded corners]{どうやって\textbf{階層構造}を作る?}
      \end{tikzpicture}
    }
  \end{center}
\end{frame}

\subsection{継承を用いた表現}
\begin{frame}[fragile]
  \frametitle{継承を用いた表現}

  \begin{center}
    \begin{tikzpicture}[scale=.8]
      \Tree[.RootException
                HttpException 
                DbException 
                [.FileException 
                    ReadException
                    WriteException ] ]
    \end{tikzpicture}
  \end{center}

  \uncover<2->{
    \lstinputlisting[style=scala]{./src/Exception.scala}
  }

  \begin{center}
    \uncover<3->{
      \begin{tikzpicture}
        \calloutquote[author={},width=8cm,position={(1.2,-0.2)},fill=blue!30,rounded corners]{どうして継承を使うの?}
      \end{tikzpicture}
    }
  \end{center}
\end{frame}

\subsection{サブタイプ多相}
\begin{frame}[fragile]
  \frametitle{サブタイプ多相}

  \begin{block}{}
    \begin{shadequote}[r]{\scriptsize 筑波大学 プログラム言語論\cite{subtype}}
      型$A$が型$B$のsubtype(部分型)のとき、型$B$の式を書くべきところに、型$A$の式を書いても良い。
    \end{shadequote}
  \end{block}

  \begin{center}
    \uncover<2->{
      \begin{tikzpicture}[scale=.8]
        \Tree[.RootException
                  HttpException 
                  DbException 
                  [.FileException 
                      ReadException
                      WriteException ] ]
      \end{tikzpicture}
    }
  \end{center}

  \uncover<3->{
    \begin{exampleblock}{例}
      \begin{itemize}
        \item<4-> RootExceptionを書くべきところにHttpExceptionを書く
        \item<5-> RootExceptionを書くべきところにReadExceptionを書く
      \end{itemize}
    \end{exampleblock}
  }
\end{frame}


\begin{frame}[fragile]
  \frametitle{階層構造の変更}

  \begin{center}
    \begin{tikzpicture}[scale=.8]
      \Tree[.RootException
                HttpException 
                DbException 
                [.FileException 
                    ReadException
                    WriteException ] ]
    \end{tikzpicture}

    \uncover<2->{
      \begin{tikzpicture}
        \calloutquote[author={},width=7cm,position={(-1.2,-0.3)},fill=green!30,rounded corners]{\textbf{後から}この階層構造を変更できる?}
      \end{tikzpicture}
    }

    \uncover<3-> {
      \begin{tikzpicture}[scale=.8]
        \Tree[.RootException
                  [.\textcolor{red}{DbAndHttpException}
                      HttpException 
                      DbException ] 
                  [.FileException 
                      ReadException
                      WriteException ] ]
      \end{tikzpicture}
    }
  \end{center}
\end{frame}

\begin{frame}[fragile]
  \begin{center}
    \begin{tikzpicture}
      \calloutquote[author={},width=5cm,position={(1.2,-0.3)},fill=red!30,rounded corners]{\huge\textbf{無理}では?}
    \end{tikzpicture}

    \uncover<2->{
      \begin{tikzpicture}
        \calloutquote[author={},width=11cm,position={(-1.2,-0.3)},fill=blue!30,rounded corners]{\huge\textbf{継承}でやるのはよくない?}
      \end{tikzpicture}
    }

    \uncover<3->{
      \begin{tikzpicture}
        \calloutquote[author={\includegraphics[width=1cm]{./img/bird2x.png}},width=8cm,position={(1.2,-0.5)},fill=green!30,rounded corners]{\huge\textbf{型クラス}でやろう!}
      \end{tikzpicture}
    }
  \end{center}
\end{frame}

\section{新しい型クラス}
\begin{frame}[fragile]
  \frametitle{新しい型クラス}

  \uncover<2->{
    \begin{block}{変換を表す型クラス\lstinline|:~>|}
      型$A$から型$B$への変換ができることを表す型クラス
      \uncover<3->{
        \lstinputlisting[style=scala]{./src/Transform.scala}
      }
      \uncover<4-> {
        \lstinline|:~>|のインスタンスとして、階層構造を定義
      }
    \end{block}
  }

  \uncover<5->{
    \begin{exampleblock}{例}
      \lstinputlisting[style=scala]{./src/Instance.scala}
    \end{exampleblock}
  }
\end{frame}

\begin{frame}[fragile]
  \frametitle{階層構造の拡張}
  
  \begin{enumerate}
    \item<2-> \textbf{新しい}エラー値を定義
    \item<3-> 型クラス\lstinline|:~>|のインスタンスを定義
    \item<4-> 型クラス\lstinline|:~>|のインスタンスを使うようにEitherを改造
  \end{enumerate}
\end{frame}

\begin{frame}[fragile]
  \frametitle{新しいエラー値の定義}

  \uncover<2->{
    \lstinputlisting[style=scala, firstline=1, lastline=1]{./src/DbAndHttpException.scala}
  }
  
  \uncover<3->{
    \begin{block}{継承に基づく階層構造}
      \begin{center}
        \begin{tikzpicture}[scale=.7]
          \tikzset{level distance=2cm}
          \Tree[.RootException
                    \textcolor{red}{DbAndHttpException}
                    HttpException
                    DbException
                    [.FileException 
                        ReadException
                        WriteException ] ]
        \end{tikzpicture}
      \end{center}
    \end{block}
  }
\end{frame}

\section*{参考文献}
\begin{frame}
  \frametitle{参考文献}

  \nocite{*}
  \bibliographystyle{junsrt_url}
  \bibliography{ref}
\end{frame}

\begin{frame}
  \centering
  {\Huge Thank you for listening!}

  \quad

  {\Huge Any question?}
\end{frame}

\end{document}
